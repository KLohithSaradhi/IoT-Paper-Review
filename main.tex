\documentclass{article}
\usepackage[utf8]{inputenc}
\usepackage{geometry}
 \geometry{
 a4paper,
 total={170mm,257mm},
 left=20mm,
 top=20mm,
 }
 \usepackage{graphicx}
 \usepackage{titling}

 \title{Integration of Cloud Computing with Internet of Things: Challenges and Open Issues
}
\author{K Lohith Saradhi}
\date{22 January 2023}
 
 \usepackage{fancyhdr}
\fancypagestyle{plain}{%  the preset of fancyhdr 
    \fancyhf{} % clear all header and footer fields
    \fancyfoot[L]{\thedate}
    \fancyhead[L]{Paper Review}
    \fancyhead[R]{\theauthor}
}
\makeatletter
\def\@maketitle{%
  \newpage
  \null
  \vskip 1em%
  \begin{center}%
  \let \footnote \thanks
    {\LARGE \@title \par}%
    \vskip 1em%
  \end{center}%
  \par
  \vskip 1em}
\makeatother

\usepackage{cmbright}

\begin{document}

\maketitle

\noindent\begin{tabular}{@{}llll}
    Student & \theauthor\ & 21110257
\end{tabular}

\section*{Why Cloud Computing and IoT are compatible?}
\begin{enumerate}
  \item IoT has devices to sense and detect activity everywhere. Cloud computing involves using computing resources from everywhere. The basic framework for integrating of these two domains is already existing due to their current intra-connectivity through the internet.
  \item The limitations of IoT (limited computational and storage ability) can be solved by the Cloud.
  \item The inflow of data from the implementations of IoT in the real world right now can help in expanding the uses of the Cloud.
\end{enumerate}

\section*{Benefits of integrating IoT into Cloud}
\begin{enumerate}
  \item Communication : Both the IoT and the Cloud computing paradigms have a very extensive network of devices in the current world. The IoT enables smooth data collection and distribution while the Cloud provides an effective and economical solution to connect, manage, and track anything by using the pre-existing apps and portals.
  \item Storage : IoT is on billions of devices, hence has a hug  number of information sources which generate enormous amounts of data. Handling this Big Data can be done by the Cloud in real time. It even provides new chances for integration, aggregation and application of this data. 
  \item Ease of access : Integrating IoT into the Cloud can bring down the deployment costs drastically making IoT accessible to anyone connected to the cloud. With the IoT being connected to Cloud indefinitely real time processing of data is feasible enabling more application innovations
\end{enumerate}

\section*{Challenges}
\begin{enumerate}
  \item Security of sensitive data which will be made available to everyone on the cloud.
  \item The heterogeneity in the type of devices and their interfaces on the IoT hinders the integration of IoT into Cloud. 
  \item Big Data handling of the scale that the IoT holds now would require an efficient, a fast enough data management system.
  \item Monitoring of the flow of data is a difficult task considering the rate and scale of the data inflow from the vast pre-existing pool of IoT devices.
\end{enumerate}

\section*{Work to be done }
\begin{enumerate}
  \item Standardisation : A standard protocol has to agreed upon for ease of communication and monitoring of data.
  \item Cloud Capabilities : Work has to done to improve security and latency of the existing Cloud frameworks. 
  \item Big Data handling: The complexity, volume and rate of data being generated poses a challenge to integration. Work has to be done to make systems handle enormous amounts data in real time
  \item Efficiency : Increase raw computing power is not enough, the scaling up of hardware should be sustainable too.
\end{enumerate}

\section*{My Views}
Integration of IoT into the Cloud could open up new paradigm of technologies and innovations. With many advances in the recent years in these verticals individually, it seems like it is the perfect time to harness this untapped potential. The integration is not easy and is riddled with challenges but the rate of advancements in both the verticals is promising. 

\section*{Agreements}
\begin{enumerate}
  \item IoT integration into Cloud is a necessary tool in the arsenal of computer science research enthusiasts.
  \item The opportunities that will be opened with this integration outweigh the initial cost of the integration.
\end{enumerate}

\section*{Pitfalls}
\begin{enumerate}
  \item The paper doesn’t discuss the benefit of this integration in other domains like Applied Science, AI and ML.
  \item The paper doesn’t specify the details of the steps that need to be taken to address the challenges faced by this integration.
\end{enumerate}
\end{document}